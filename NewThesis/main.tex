%%%%%%%%%%%%%%%%%%%%%%%%%%%%%%%%%%%%%%%%%%%%%%%%%%%%%%%%%%%%%%%%%%%%%%%%%%
%%        file: atprog-thesis-template-eng-20190426.tex
%%      author: Heiki Kasemgi <heiki.kasemagi@eesti.ee>
%% description: tex source
%% computer engineering
%% thesis template
%% copyright (c) 2015-2019, Heiki Kasemägi, <heiki.kasemagi@eesti.ee>
%%
%% 297x210 format A4
%%
%% double "%" marks a comment
%% single "%" marks a commented command
%% the complete set of (commented) commands is between the lines marked as 
%% "%%=== title ===" and "%%%%%%%%%"
%%%%%%%%%%%%%%%%%%%%%%%%%%%%%%%%%%%%%%%%%%%%%%%%%%%%%%%%%%%%%%%%%%%%%
%% This program is free software; you can redistribute it and/or
%% modify it under the terms of the GNU General Public License
%% as published by the Free Software Foundation; either version 2
%% of the License, or (at your option) any later version.
%%
%% This program is distributed in the hope that it will be useful,
%% but WITHOUT ANY WARRANTY; without even the implied warranty of
%% MERCHANTABILITY or FITNESS FOR A PARTICULAR PURPOSE.  See the
%% GNU General Public License for more details.
%%
%% You should have received a copy of the GNU General Public License
%% along with this program; if not, write to the Free Software
%% Foundation, Inc., 51 Franklin Street, Fifth Floor, Boston,
%% MA  02110-1301, USA
%%
%%%%%%%%%%%%%%%%%%%%%%%%%%%%%%%%%%%%%%%%%%%%%%%%%%%%%%%%%%%%%%%%%%%%%%%%%%
%% Käesolev programm on vaba tarkvara. Te võte seda edasi levitada ja/või
%% muuta vastavalt GNU Üldise Avaliku Litsentsi tingimustele, nagu need on
%% Vaba Tarkvara Fondi poolt avaldatud; kas Litsentsi versioon number 2
%% või (vastavalt Teie valikule) ükskõik milline hilisem versioon.
%%
%% Seda programmi levitatakse lootuses, et see on kasulik, kuid ILMA
%% IGASUGUSE GARANTIITA; isegi KESKMISE/TAVALISE KVALITEEDI GARANTIITA või
%% SOBIVUSELE TEATUD KINDLAKS EESMRGIKS. Üksikasjade suhtes vaata GNU
%% Üldist Avalikku Litsentsi
%%
%% Te peaks olema saanud GNU Üldise Avaliku Litsentsi koopia koos selle
%% programmiga, kui ei, siis kontakteeruge Free Software Foundation'iga,
%% 59 Temple Place - Suite 330, Boston, MA 02111-1307, USA
%%
%%%%%%%%%%%%%%%%%%%%%%%%%%%%%%%%%%%%%%%%%%%%%%%%%%%%%%%%%%%%%%%%%%%%%%%%%%

\documentclass[12pt]{report}
\usepackage{a4}
%\usepackage{fancyhdr}
\usepackage[nottoc]{tocbibind} % option removes the "Contents" form the
% contents
\usepackage{times}
\usepackage{cite}
\usepackage{graphicx}
\usepackage{rotating}
\usepackage{hyperref}
\usepackage[nopar]{lipsum} % package for dummmy texts
\usepackage{titlesec}
%\usepackage{ccaption}
%%=== B5 page setup ===
%\setlength{\paperwidth}{17.5cm} %B5
%\setlength{\paperheight}{25cm} %B5
%\setlength{\textwidth}{12.5cm} %B5
%\setlength{\textheight}{19.5cm} %B5
%%%%%%%%%%%%%%%%%%%%%%%%%%%%%%%%%%%%
%%=== A4 page setup ===
\setlength{\paperwidth}{21.0cm} 
\setlength{\paperheight}{29.7cm}
\setlength{\textwidth}{16cm}
\setlength{\textheight}{25cm}
%%%%%%%%%%%%%%%%%%%%%%%%%%%%%%%%%%%
%%=== common ===
\setlength{\leftmargin}{2.5cm} %2.5cm
\setlength{\rightmargin}{2.5cm} %2.5cm
\setlength{\oddsidemargin}{0cm}
\setlength{\evensidemargin}{0cm}
\setlength{\topmargin}{-2.0cm} %-2.0cm
%%\setlength{\headsep}{0.0cm}
%%\setlength{\headheight}{0.0cm}
\setlength{\footskip}{1.0cm} %-2.5cm
\setlength{\topskip}{0.5cm}
%%%%%%%%%%%%%%%%%%%%%%%%%%%%%%%%%%%%%%
%%=== watermark setup ===
%\usepackage[outline,light,portrait]{draftcopy}
%\draftcopyName{Heiki Kasemagi Ph. D. Thesis}{45}
%\draftcopyPageX{30}
%\draftcopyPageY{50}
%\draftcopyBottomX{0}
%\draftcopyBottomY{-100}
%%%%%%%%%%%%%%%%%%%%%%%%%%%%
%\renewcommand{\thefigure}{\arabic{figure}}
%\renewcommand{\figurename}{Figure}
%\setcounter{figure}{0}
%%=== 1.5 row spacing ===
%\renewcommand{\baselinestretch}{1.5}
%%%%%%%%%%%%%%%%%%%%%%%%%%%%%%
%%=== disabling the hypenation ===
%\hyphenpenalty=1000
%%%%%%%%%%%%%%%%%%%%%%%%%%%%%%
%%\renewcommand{\contentsname}{\protect \center{Contents}}
%%\renewcommand{\listfigurename}{\protect \centering List of Figures}
%%\renewcommand{\chaptername}{\protect \centering Chapter}
%%=== translations to Estonian ===
%%=== comment for English ===
%%%\renewcommand{\bibname}{References}
%%%\renewcommand{\contentsname}{Content}
%%%\renewcommand{\listfigurename}{Jooniste loetelu}
%%%\renewcommand{\listtablename}{Tablelite loetelu}
%\renewcommand{\chaptername}{}
%%%\renewcommand{\figurename}{Joonis}
%%%\renewcommand{\tablename}{Tabel}
%%%%%%%%%%%%%%%%%%%%%%%%%%%%%%%%
%%=== ===
%%\makeatletter
%%\renewcommand{\section}{\@startsection{section}{1}{0mm}
%%{\baselineskip}%
%%{\baselineskip}{\Large\centering}}%
%%\makeatother
%%%%%%%%%%%%%%%%%%%%%%%%%%%%%%%%

%%%========== make own fig caption not listed in LOF ==========
\newcommand\myCaption[1]{\normalsize\refstepcounter{figure}%
   \figurename\ \thefigure :\ #1}
%%%============================================================

%%%========== remove page number for TOC ==========
%\makeatletter
%\let\myTOC\tableofcontents
%\renewcommand{\tableofcontents}{%
%  \begingroup
%  \let\ps@plain\ps@empty
%  \pagestyle{empty}
%  \myTOC
%%  \clearpage
%  \endgroup%
%}
%\makeatother
%%%================================================
%%=== insert figure ===
%% arguments: 
%% #1 - figure location in the page, recommended "h"
%% #2 - figure width, recommended "x.x\textwidth" where x.x is a fraction
%%      of the page width, e.g. "0.54\textwidth" is 54 % of the text width
%% #3 - figure path
%% #4 - figure caption
%% #5 - figure label for referencing in the text
\newcommand\insertfigure[5]{
\begin{figure}[#1]
\begin{center}
\includegraphics[width=#2]{#3}
\end{center}
\caption{#4}
\label{#5}
\end{figure}
}
%%%%%%%%%%%%%%%%%%%%%%%%%%%%%%%%%%%%%%%%%%%
%%=== modified chapter title ===
\newcommand{\hsp}{\hspace{20pt}}
\titleformat{\chapter}[hang]{\Huge\bfseries}{\thechapter\hsp}{0pt}{\Huge\bfseries}
%%%%%%%%%%%%%%%%%%%%%%%%%%%%%%%%%%%%%%%%%%%

\begin{document}
%\noindent
\thispagestyle{empty}
%%=== titlepage ===
\begin{large}
\begin{center}
\vspace{20mm}
Tartu University
\\[5mm]
Faculty of Science and Technology
\\[5mm]
Institute of Technology
\\[5mm]
\vspace{50mm}
Given\_name Surname
\\[10mm]
\textbf{My Thesis title}
\\[10mm]
Master's thesis (30 EAP)\\
Robotics and Computer Engineering\\
%Magistrit\"o\"o (30 EAP)\\
\end{center}
\vspace{20mm}
\flushright{Supervisor:}
%\flushright{Juhendajad:}
\\[5mm]
title Given\_name Surname\\
%amet Eesnimi Perekonnanimi\\
%%...
\begin{center}
%% adjust vspace value according to title rows and supervisor count
\vspace{65mm}
Tartu 2019\\
\end{center}
\end{large}
\newpage
%%=== Abstract ===
\chapter*{Res\"umee/Abstract}
\addcontentsline{toc}{chapter}{Res\"umee/Abstract}
\textbf{Minu l\~oput\"o\"o pealkiri}
\\[5mm]
Res\"umee ehk abstrakt on ``kokkuv\~otlik, oluliste seisukohtade ja v\"aidete \"ulevaatlik esitus[..].
\\[5mm]
Eesm\"ark on v\~oimalikult t\"apselt ja l\"uhidalt edasi anda teksti sisu ja selles esitatud peamised v\"aited. Oluline on faktiline korrektsus (ei lisata midagi, mida tekst ei toeta) ja k\~oige olulisemate seisukohtade esiletoomine.'' \cite{bib:wiki-abstract}.
\\[5mm]
K\"aesoleval juhul peaks abstrakt andma kondenseeritud \"ulevaate kogu tekstist, sealhulgas ka olulisematest tulemustest, sest see tekst kantakse \"ule Tartu \"Ulikooli raamatukogu elektrooniliste materjalide hoidlasse \href{http://dspace.utlib.ee/dspace/}{DSpace'i}.
\\[5mm]
\textbf{CERCS:} T120 S\"usteemitehnoloogia, arvutitehnoloogia; T125 Automatiseerimine, robootika, control engineering; (n\"aidis: muuda, t\"aienda vastaval oma t\"o\"o sisule \cite{cercs})
\\[5mm]
\textbf{M\"arks\~onad:} arvutid, kontroll, robootika (n\"aidis: muuda, t\"aienda vastaval oma t\"o\"o sisule)
\\[10mm]
\textbf{My thesis title}
\\[5mm]
Abstract is ``a concise overview of important postitions and statements[..].
\\[5mm]
The aim is to convey the content of the thesis and the main statements contained therein as precisely and briefly as possible. Important is factual correctness (not adding something that is not supported by the text) and highlighting all important views.'' \cite{bib:wiki-abstract}. 
\\[5mm]
\textbf{CERCS:} T120 Systems engineering, computer technology; T125 Automation, robotics, control engineering (an example: modifiy, complement according to the content of you thesis \cite{cercs})
\\[5mm]
\textbf{Keywords:} computers, control, robotics (an example: modifiy, complement according to the content of you thesis)
%%=== let's add the table of contents, list of figures and tables ===
\tableofcontents
\listoffigures
\listoftables
%%=== list of abbreviations, constants etc. ===
\chapter*{L\"uhendid, konstandid, m\~oisted}
\addcontentsline{toc}{chapter}{Abbreviations. Constants. Generic Terms}
\begin{description}
\item[ROS] - Robot Operating System
\item[c] - electromagnetic wave propagation speed in vacuum
\end{description}
%%=== chapter: introduction ===
\chapter{Introduction}

The introduction is a place where you put your problem into the so-called world context, the question is asked why is it necessary to investigate this problem. \cite{bib:tyypviide,bib:tyypeng1,bib:rand}.
\\[5mm]
\lipsum[5]

%%=== section: problem overview ===
\section{Problem Statement}
Give the overview and the essence of the problem \cite{bib:wiki-abstract,bib:rand}.
\\[5mm]
\lipsum[7]

%%=== section: goals ===
\section{Objectives and Roadmap}
Define the objectives of the thesis. Also describe the roadmap to achive the goals.
\\[5mm]
\lipsum[8]
\\[5mm]
\lipsum[9]

%%=== chapter: problem review ===
\chapter{State of the Art}
In this chapter you describe the most recent and relevant achievements in the field and in the thesis context based on the publications. Also, this overview should lead to the understanding of the stimuli of the thesis and make clear why the previous studies motivated the current research.
%%=== section: experiments ===
\section{Experimental}
The overview can include experimental results.
\\[5mm]
\lipsum[10]
%%=== section: theory, simulations ===
\section{Theoretical solutions. Simulations}
The overview can include therotical solutions, simulation results etc.
\\[5mm]
\lipsum[11]
%%=== chapter: methodology ===
\chapter{Methodology}
In this chapter you describe the object you study, the reserach method(s) you use for the research and analysis, also research tools, materials etc.
%%=== section: methods ===
\section{Research Methods}
Physical laws are usually described by the formulas:
\\[5mm]
%%=== formula example ===
\begin{equation}
\label{eq:dif}
\left \{
\begin{array}{ll}
e_{a}\frac{\partial^{2}u}{\partial t^{2}} + d_{a}\frac{\partial u}{\partial t} + \nabla \cdot (-c\nabla u - \alpha u + \gamma) + \beta \cdot \nabla u + au = f & piirkonnas\quad \Omega\\
\mathbf{n}\cdot (c\nabla u +\alpha u -\gamma) + qu = -g-h^{T}\mu & rajal\quad \partial\Omega\\
hu = r & rajal\quad \partial\Omega
\end{array} \right .
\end{equation}
%%%%%%%%%%%%%%%%%%%%%%%%%%%
\\[5mm]
This formula \textbf{\ref{eq:dif}} is a partial differential equation with boundary conditons.
\\[5mm]
\lipsum[12]
%%=== section: hardware ===
\section{Research Tools}
\lipsum[13]
%%=== section: object ===
\section{Research Object}
\lipsum[14]
%%=== section: materials ===
\section{Research Materials}
\lipsum[21]
%%=== chapter: results ===
\chapter{The Results}
In this chapter you describe how did you reach the results and present the results.
%%=== section: results 1 ===
\section{Results in the First Method}
\lipsum[15]
%%=== section: results 2 ===
\section{Results in the Second Method}
\lipsum[16]
%%=== section: results 3 ===
\section{Results in the Third Method}
These result are shown in Fig. \textbf{\ref{fig:randomimage14}}.
\\[5mm]
\lipsum[17]
%%=== figure example ===

%%%%%%%%%%%%%%%%%%%%%%%%%
%%=== chapter: analysis ===
\chapter{Analysis and Discussion}

In this chapter you analyse and discuss the result obtained in different methods, do the comparison and draw the conclusions.
%%=== section: analysis 1 ===
\section{Analysis of the Method One}
The result can be presented as a table \textbf{\ref{tab:models}}.
\\[5mm]
%%=== table example ===
\begin{table}[h]
\caption{Simulated Models}
\label{tab:models}
\begin{center}
\begin{tabular}{l | c r l c c c}
Type & Simulation box & EO & Salt & Li:EO & Particle & Temp. /K\\
& /{\AA}x{\AA}x{\AA} & & & diameter /{\AA}\\
\hline
\hline
A01 & 26x21x22  & 200 & --         & --   & --   & 360      \\
A02 & 26x21x22  & 200 & LiCl       & 1:10 & --   & 360      \\
A03 & 26x21x22  & 200 & LiBr       & 1:10 & --   & 360      \\
A04 & 26x21x22  & 200 & LiI        & 1:10 & --   & 360      \\
A05 & 31x31x31  & 455 & --         & --   & 14   & 360      \\
A06 & 31x31x31  & 455 & --         & --   & 14   & 360      \\
A07 & 31x31x31  & 455 & --         & --   & 14   & 360      \\
A08 & 31x31x31  & 455 & LiCl       & 1:10 & 14   & 360      \\
A09 & 31x31x31  & 455 & LiBr       & 1:10 & 14   & 360      \\
A10 & 31x31x31  & 455 & LiI        & 1:10 & 14   & 360      \\
A11 & 37x37x37  & 787 & --         & --   & 18   & 360      \\
A12 & 37x37x37  & 787 & --         & --   & 18   & 360      \\
A13 & 37x37x37  & 787 & --         & --   & 18   & 360      \\
A14 & 37x37x37  & 787 & LiCl       & 1:10 & 18   & 360      \\
A15 & 37x37x37  & 787 & LiBr       & 1:10 & 18   & 360      \\
A16 & 37x37x37  & 787 & LiI        & 1:10 & 18   & 360      \\
\\
B01 & 24x24x24  & 200 & LiBF$_{4}$ & 1:20 & --   & 293      \\
B02 & 31x31x31  & 455 & LiBF$_{4}$ & 1:20 & 14   & 293      \\
B03 & 14x14x200 & 294 & LiBF$_{4}$ & 1:20 & slab & 293      \\
\\
C01 & 28x22x24  & 200 & LiCl       & 1:20 & --   & 290, 330 \\
C02 & 28x22x24  & 200 & LiCl       & 1:35 & --   & 290, 330 \\
C03 & 28x22x24  & 200 & LiCl       & 1:50 & --   & 290, 330 \\
C04 & 33x33x33  & 455 & LiCl       & 1:20 & 14   & 290, 330 \\
C05 & 33x33x33  & 455 & LiCl       & 1:35 & 14   & 290, 330 \\
C06 & 33x33x33  & 455 & LiCl       & 1:50 & 14   & 290, 330 \\
\hline
\hline
\end{tabular}
\end{center}
\end{table}
%%%%%%%%%%%%%%%%%%%%%%%%%%%%
\lipsum[18]
%%=== section: analysis 2 ===
\section{Analysis of the Method Two}
\lipsum[19]
%%=== section: analysis conclusions ===
\section{Discussion of the results}
\lipsum[1]
%%=== chapter: summary (original language) ===
\chapter{Conclusion}
\addcontentsline{toc}{chapter}{Conclusion. Future Perspectives}
The conlusion in the part, where you conclude if you did reach to objectives set up in the beginning of the thesis, if the results did confirm the hypothesis or if the device you designed was built and functioned as expected. You bring out and formulate all the major results and the relations betweem them. No new data is presented.
\\[5mm]
Also, you can present the ideas, how to continue the results, is it's possible at all or essential.
\lipsum[20]
%%=== chapter: acknowledgements ===
\chapter*{Acknowledgements}

Acknowledge everybody you feel for the help, support etc.
\\[5mm]
It would be nice to add a scanned image of you handwritten signature after the acknowledgements.

%%=== references ===
\begin{thebibliography}{99}
%% reference to the network ressource, namely Wikipedia according to its recommendations
\bibitem{bib:wiki-abstract} Vikipeedia, vaba ents\"uklopeedia 2015. - Res\"umee.\\ \href{http://et.wikipedia.org/w/index.php?title=Res\%C3\%BCmee\&oldid=4093046}{http://et.wikipedia.org/w/index.php?title=Res\%C3\%BCmee\&oldid=4093046} 7.04.2015, 07:14 (UTC).
\bibitem{cercs} Common European Research Classification Scheme (CERCS) Teadusvaldkondade ja -erialade klassifikaator \href{https://www.etis.ee/Portal/Classifiers/Details/d3717f7b-bec8-4cd9-8ea4-c89cd56ca46e}{https://www.etis.ee/Portal/Classifiers/Details/d3717f7b-bec8-4cd9-8ea4-c89cd56ca46e} (ETIS); PDF: \href{https://wiki.ut.ee/download/attachments/16581162/Common\%20European\%20Research\%20Classification\%20Scheme.pdf}{https://wiki.ut.ee/download/attachments/16581162/Common\%20European\\
\%20Research\%20Classification\%20Scheme.pdf}
%% common scientific paper reference template
\bibitem{bib:tyypviide}
Gname1. Surname1, Gname2. Surname2 ja Gname3. Surname3, ``Papaer Title'', \textit{Journal Title}, \textbf{volume}, year, first\_page-last\_page, DOI:number.
%% common paper example
\bibitem{bib:tyypeng1}
M. Winter, J. O. Besenhard, M. E. Spahr, and P. Nov\'ak,
``Insertion electrode materials for rechargeable lithium batteries'',
\textit{Advanced Materials}, 1998, \textbf{10}, 725--763, DOI:\href{http://dx.doi.org/10.1002/(SICI)1521-4095(199807)10:10<725::AID-ADMA725>3.0.CO;2-Z}{10.1002/(SICI)1521-4095(199807)10:10<725::AID-ADMA725>3.0.CO;2-Z}.
%% common paper example
\bibitem{bib:rand}
R. M. Dell and D. A. J. Rand,
``Energy storage --- a key technology for global energy sustainability'',
\textit{J. Power Sources}, 2001, \textbf{100}, 2--17, DOI:\href{http://dx.doi.org/10.1016/S0378-7753(01)00894-1}{10.1016/S0378-7753(01)00894-1}.
%% book reference
\bibitem{bib:book}
I. T. Masters, \textit{Practical Neural Network Recipes in C++}, Academic, New York, 1993.
%% papre or chapter reference in a book
\bibitem{bib:book-chap}
I. B. L. Shoop, A. H. Sayles, and D. M. Litynski, ``New devices for optoelectronics: smart pixels,'' in \textit{Handbook of Fiber Optic Data Communications}, C. DeCusatis, D. Clement, E. Maass, and R. Lasky, eds., Academic, San Diego, Calif., 1997, pp. 705-758.
%% published conference presentations
\bibitem{bib:conf}
I. R. E. Kalman, ``Algebraic aspects of the generalized inverse of a rectangular matrix,'' in \textit{Proceedings of Advanced Seminar on Genralized Inverse and Applications}, M. Z. Nashed, ed., Academic, San Diego, Calif., 1976, pp. 111-124.
%% unpublished presentations
\bibitem{bib:conf-unpub-1}
D. Steup and J. Weinzierl, ``Resonant THz-meshes,'' presented at the Fourth International Workshop on THz Electronics, Erlangen-Tennenlohe, Germany, 5-6 Sept. 1996.
\bibitem{bib:conf-unpub-2}
D. Steup and J. Weinzierl, ``Resonant THz-meshes,'' Abstracts of the Fourth International Workshop on THz Electronics, Erlangen-Tennenlohe, Germany, 5-6 Sept. 2004, Academic, San Diego, 2005, p. 111-112.
%% accepted paper
\bibitem{bib:accept}
I. D. Piao, Q. Zhu, N. K. Dutta, S. Yan, and L. L. Otis, ``Cancelation of coherent artifacts in optical coherence tomography imaging,'' \textit{Appl. Opt.}, accepted for publication.
\end{thebibliography}
%%=== appendixes ===
\chapter*{Appendices}
\addcontentsline{toc}{chapter}{Lisad}
\newpage
%%=== common license ===
\chapter*{Non-exclusive licence to reproduce thesis and make thesis public}
\addcontentsline{toc}{chapter}{Non-exclusive license}
\thispagestyle{empty}
I, Given\_name Surname
\begin{enumerate}
\item herewith grant the University of Tartu a free permit (non-exclusive licence) to reproduce, for the purpose of preservation, including for adding to the DSpace digital archives until the expiry of the term of copyright,
\begin{center}
\textbf{``My thesis title''}
\end{center}
supervised by Given\_name Surname
\item I grant the University of Tartu a permit to make the work specified in p. 1 available to the public via the web environment of the University of Tartu, including via the DSpace digital archives, under the Creative Commons licence CC BY NC ND 3.0, which allows, by giving appropriate credit to the author, to reproduce, distribute the work and communicate it to the public, and prohibits the creation of derivative works and any commercial use of the work until the expiry of the term of copyright. 
\item I am aware of the fact that the author retains the rights specified in p. 1 and 2.
\item I certify that granting the non-exclusive licence does not infringe other persons' intellectual property rights or rights arising from the personal data protection legislation.
\end{enumerate}
\vspace{2cm}
\textit{Author's name}
\\
\textbf{DD.MM.YYYY}
\end{document}
